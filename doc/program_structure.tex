\section{Структура программы}

Epicf-v0.01 написан на C. 

Основная структура в программе - 'domain'.
Идея в том, что она должна хранить в себе всю информацию, необходимую для 
моделирования. Эту информацию можно разделить на 3 части: информация о 
временной сетке, о пространственной области и ее дискретизации, и информация о частицах.

Генерация частиц:
Максвелловское распределение по компонентам импульса - произведение
нескольких нормальных распределений. Используется функция \texttt{gsl\_ran\_gaussian} [ссылка]
из библиотеки GSL, которая выдает случайную величину с распределением вероятности
\begin{gather}
  p(x) dx = \frac{1}{\sqrt{2 \pi \sigma^2}} e^{-x^2 / 2\sigma^2} dx
\end{gather}

\todo{(в программе вроде пока один генератор на обе компоненты. эквивалентно?) }

Интерполяция зарядов на узлы сетки. Форма частиц. Картинка. 

Уравнение Пуассона:
\begin{gather}
  \Delta \phi = - 4 \pi \rho  
  \\
  \frac{ \phi_{i-1,j} - 2 \phi_{i,j} + \phi_{i+1,j} }{ \Delta x^2 }
  + 
  \frac{ \phi_{i,j-1} - 2 \phi_{i,j} + \phi_{i,j+1} }{ \Delta y^2 }
  =
  -4 \pi \rho_{i,j}
\end{gather}
\todo{Дописать про дискретизацию и граничные условия.}
Для численного решения используется библиотека FISHPACK [ссылка]. 
При вызове фортрановской функции из C нужно учитывать особенности 
представления многомерных массивов в памяти [ссылка].
В фортране - column-major order, в C - row-major. 
В программе за преобразование между двумя форматами отвечают функции \todo{...}.

Расчет полей по потенциалу:
\begin{gather}
  \vec{ E } = - \nabla \phi
  \\
  E_{x_i} =  - \frac{ \phi_{i+1,j} - \phi_{i-1,j} }{ 2 \Delta x }  - \mbox{ центральная разность }
  \\
  E_{x_i} =  - \frac{ \phi_{i+1,j} - \phi_{i,j} }{ \Delta x } - \mbox{ на границе }
\end{gather}

Обновление импульсов и координат частиц:
\begin{gather}
  \frac{ d \vec{p} }{ d t } = \vec{ F } = q \vec{ E }
  \\
  \frac{ d \vec{r} }{ d t } = \frac{ \vec{p} }{ m }
\end{gather}
Используется схема Leap-frog
\begin{gather}
  \vec{p}_{t+1/2} = \vec{p}_{t-1/2} + q \vec{E} \Delta t  
  \\
  \vec{r}_{t+1} = \vec{r}_{t} + \frac{ \vec{p}_{t+1/2} }{ m } \Delta t
\end{gather}

%%% Local Variables:
%%% mode: latex
%%% TeX-master: "epicf"
%%% End:
